\documentclass[11pt, a4paper]{article}
\usepackage[T1]{fontenc}
\usepackage{lmodern}
\usepackage[UKenglish]{babel}
\usepackage{csquotes}
\usepackage{graphicx}
\usepackage{hyperref}
\usepackage{amsmath}
\usepackage{amssymb}
\usepackage{booktabs}
\usepackage{longtable}
\usepackage{xcolor}
\usepackage{listings}
\usepackage{enumitem}
\usepackage[a4paper, margin=12mm]{geometry}
\frenchspacing

% remove page numbering
\pagenumbering{gobble}

% uniform spacing after punctuation
\frenchspacing

% no indent
\usepackage[parfill]{parskip}

% custom commands
\newcommand{\code}[1]{\texttt{#1}}

% code block configuration
\lstset{
    basicstyle=\ttfamily\small,
    commentstyle=\ttfamily\upshape,
    breaklines=true,
    frame=single,
    xleftmargin=2em,
    framexleftmargin=1.5em,
    showstringspaces=false,
    extendedchars=false,
    upquote=true,
    literate={--}{{-{}-}}2,
    language={}
}

\begin{document}

\begin{center}
    {\LARGE\textbf{Essential C Syntax for CUDA Kernels}}\\[0.5em]
    {\large A Reference Guide for Python Programmers}
\end{center}

\vspace{1em}

\section*{Introduction}

CuPy raw kernels allow you to write custom GPU operations using CUDA C code
embedded as strings within Python. Understanding basic C syntax is essential
because:

\begin{itemize}
    \item GPUs execute compiled C/C++ code, not Python.
    \item Raw kernels provide finer control over parallel computation.
    \item Custom operations can be optimised for specific data patterns.
\end{itemize}

This guide covers the essential C syntax needed for writing CUDA kernels in
CuPy.

\section*{Basic Data Types}

\subsection*{C Types and NumPy Equivalents}

\begin{center}
\begin{tabular}{llll}
\toprule
C Type & NumPy Equivalent & Size (bytes) & Range/Notes \\
\midrule
\code{char} & \code{np.int8} & 1 & -128 to 127 \\
\code{unsigned char} & \code{np.uint8} & 1 & 0 to 255 \\
\code{int} & \code{np.int32} & 4 & $-2^{31}$ to $2^{31}-1$ \\
\code{unsigned int} & \code{np.uint32} & 4 & 0 to $2^{32}-1$ \\
\code{float} & \code{np.float32} & 4 & \textasciitilde{}7 decimal digits \\
\code{double} & \code{np.float64} & 8 & \textasciitilde{}15 decimal digits \\
\bottomrule
\end{tabular}
\end{center}

\subsection*{Numeric Literals}

\begin{lstlisting}[language=C]
// floating-point suffixes
float x = 1.0f;     // f suffix for float
double y = 1.0;     // no suffix defaults to double

// integer suffixes
unsigned int a = 10u;    // u suffix for unsigned
long long b = 100ll;     // ll suffix for long long
\end{lstlisting}

\section*{Variable Declaration}

\subsection*{C Declaration Syntax}

\begin{lstlisting}[language=C]
// C requires type declaration
int x;              // declaration without initialisation
int y = 5;          // declaration with initialisation
float z = 3.14f;    // float literal needs 'f' suffix

// multiple declarations
int a, b, c;        // declare multiple variables
int d = 1, e = 2;   // initialise multiple variables
\end{lstlisting}

\subsection*{Python vs C Comparison}

\begin{center}
\begin{tabular}{ll}
\toprule
Python & C \\
\midrule
\code{x = 5} & \code{int x = 5;} \\
\code{y = 3.14} & \code{float y = 3.14f;} \\
\code{name = "gpu"} & \code{char name[] = "gpu";} \\
\bottomrule
\end{tabular}
\end{center}

\section*{Operators}

\subsection*{Arithmetic Operators}

\begin{center}
\begin{tabular}{lll}
\toprule
Operator & Description & Example \\
\midrule
\code{+} & Addition & \code{a + b} \\
\code{-} & Subtraction & \code{a - b} \\
\code{*} & Multiplication & \code{a * b} \\
\code{/} & Division & \code{a / b} \\
\code{\%} & Modulo (remainder) & \code{a \% b} \\
\bottomrule
\end{tabular}
\end{center}

\textbf{Important:} Integer division truncates in C:
\begin{lstlisting}[language=C]
int result = 7 / 2;             // result is 3, not 3.5
float precise = 7.0f / 2.0f;    // result is 3.5f
\end{lstlisting}

\subsection*{Comparison Operators}

\begin{center}
\begin{tabular}{lll}
\toprule
Operator & Description & Example \\
\midrule
\code{==} & Equal to & \code{a == b} \\
\code{!=} & Not equal to & \code{a != b} \\
\code{<} & Less than & \code{a < b} \\
\code{>} & Greater than & \code{a > b} \\
\code{<=} & Less than or equal & \code{a <= b} \\
\code{>=} & Greater than or equal & \code{a >= b} \\
\bottomrule
\end{tabular}
\end{center}

\subsection*{Logical Operators}

\begin{center}
\begin{tabular}{lll}
\toprule
Operator & Description & Python Equivalent \\
\midrule
\code{\&\&} & Logical AND & \code{and} \\
\code{||} & Logical OR & \code{or} \\
\code{!} & Logical NOT & \code{not} \\
\bottomrule
\end{tabular}
\end{center}

\begin{lstlisting}[language=C]
// C logical operations
if (x > 0 && x < 10) { }     // both conditions must be true
if (x < 0 || x > 100) { }    // either condition can be true
if (!done) { }               // if not done
\end{lstlisting}

\section*{Control Flow Structures}

\subsection*{If/Else Statements}

\begin{lstlisting}[language=C]
// basic if statement
if (x > 0) {
    y = x * 2;
}

// if-else
if (temperature > 30.0f) {
    state = 1;  // hot
} else if (temperature > 20.0f) {
    state = 2;  // warm
} else {
    state = 3;  // cold
}
\end{lstlisting}

\subsection*{For Loops}

\begin{lstlisting}[language=C]
// standard for loop
for (int i = 0; i < n; ++i) {
    array[i] = i * 2;
}
\end{lstlisting}

\textbf{Note:} In CUDA kernels, parallel threads replace traditional loops.

\section*{Array Access and Pointers}

\subsection*{Basic Array Indexing}

\begin{lstlisting}[language=C]
// array access is similar to python
float value = array[index];
array[index] = new_value;

// 2d array access (linearised)
int row = 2, col = 3, width = 10;
float element = array[row * width + col];
\end{lstlisting}

\subsection*{Pointer Basics}

\begin{lstlisting}[language=C]
// in cuda kernels, arrays are passed as pointers
__global__ void kernel(float* input, float* output) {
    int idx = threadIdx.x + blockIdx.x * blockDim.x;
    output[idx] = input[idx] * 2.0f;
}
\end{lstlisting}

\section*{Type Casting}

\begin{lstlisting}[language=C]
// explicit type casting
int a = 5;
float b = (float)a;         // cast int to float

// implicit casting (be careful!)
float x = 5;                // int 5 becomes 5.0f
int y = 3.14f;              // float 3.14f becomes 3

// common casting scenarios
int idx = (int)floorf(x);   // float to int with floor
float ratio = (float)a / (float)b;  // ensure float division
\end{lstlisting}

\section*{Common Pitfalls for Python Programmers}

\subsection*{1. Semicolons Are Required}
\begin{lstlisting}[language=C]
int x = 5;    // correct
int y = 10    // error: missing semicolon
\end{lstlisting}

\subsection*{2. Variable Declaration Is Mandatory}
\begin{lstlisting}[language=C]
// python: x = 5
// C: must declare type
int x = 5;
\end{lstlisting}

\subsection*{3. Array Bounds Not Checked}
\begin{lstlisting}[language=C]
float arr[10];
float value = arr[20];    // undefined behaviour!
\end{lstlisting}

\subsection*{4. Integer Division Truncates}
\begin{lstlisting}[language=C]
int result = 7 / 2;           // result is 3
float correct = 7.0f / 2.0f;  // result is 3.5f
\end{lstlisting}

\subsection*{5. Boolean Type Differences}
\begin{lstlisting}[language=C]
// C uses 0 for false, non-zero for true
if (x) { }     // true if x != 0
if (!x) { }    // true if x == 0
\end{lstlisting}

\section*{Practical CUDA Kernel Examples}

\subsection*{Example 1: Add Constant}
\begin{lstlisting}[language=C]
__global__ void add_constant(float* data, float constant,
                            int n) {
    int idx = threadIdx.x + blockIdx.x * blockDim.x;
    if (idx < n) {
        data[idx] = data[idx] + constant;
    }
}
\end{lstlisting}

\subsection*{Example 2: Element-wise Multiplication}
\begin{lstlisting}[language=C]
__global__ void multiply_arrays(float* a, float* b,
                               float* result, int n) {
    int idx = threadIdx.x + blockIdx.x * blockDim.x;
    if (idx < n) {
        result[idx] = a[idx] * b[idx];
    }
}
\end{lstlisting}

\subsection*{Example 3: 2D Array Access}
\begin{lstlisting}[language=C]
__global__ void process_2d(float* input, float* output,
                          int width, int height) {
    int x = threadIdx.x + blockIdx.x * blockDim.x;
    int y = threadIdx.y + blockIdx.y * blockDim.y;

    if (x < width && y < height) {
        int idx = y * width + x;
        output[idx] = input[idx] * 2.0f;
    }
}
\end{lstlisting}

\end{document}
