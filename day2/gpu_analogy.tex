\documentclass[11pt, a4paper]{article}
\usepackage[T1]{fontenc}
\usepackage{lmodern}
\usepackage[UKenglish]{babel}
\usepackage{csquotes}
\usepackage{graphicx}
\usepackage{hyperref}
\usepackage{amsmath}
\usepackage{amssymb}
\usepackage{booktabs}
\usepackage{longtable}
\usepackage{xcolor}
\usepackage{listings}
\usepackage{enumitem}
\usepackage[a4paper, margin=2cm]{geometry}

% remove page numbering
\pagenumbering{gobble}

% uniform spacing after punctuation
\frenchspacing

% configure listings for code
\lstset{
    basicstyle=\ttfamily\small,
    breaklines=true,
    frame=single,
    numbers=left,
    numberstyle=\tiny,
    xleftmargin=2em,
    framexleftmargin=1.5em
}

% document metadata
\title{The Mega Restaurant: Understanding GPU Parallel Processing}
\author{}
\date{}

\begin{document}

\maketitle

\section*{Welcome to MegaBites Restaurant}

Imagine the world's largest restaurant with \textbf{1000 tables}, all occupied by hungry customers who have ordered simultaneously. Everyone expects prompt service. How can we efficiently serve them all?

\subsection*{Option 1: The CPU Approach - \enquote{The Super-Waiter}}
\begin{itemize}
    \item ONE exceptionally fast waiter
    \item Runs between all 1000 tables sequentially
    \item Takes one order, rushes to kitchen, delivers food, proceeds to next table
    \item Even at superhuman speed: 1000 tables × 2 minutes each = \textbf{over 33 hours}
    \item Customers at table 1000 would experience unacceptable waiting times
\end{itemize}

\subsection*{Option 2: The GPU Approach - \enquote{The Parallel Service Team}}
\begin{itemize}
    \item 1000 waiters, each assigned to ONE specific table
    \item All waiters work simultaneously (parallel processing)
    \item Each follows identical service procedures
    \item All 1000 tables served in just \textbf{2 minutes}
    \item Every customer receives prompt, efficient service
\end{itemize}

\section*{The Computing Connection}

\begin{table}[h!]
\centering
\begin{tabular}{ll}
\toprule
Restaurant Element & GPU Computing Equivalent \\
\midrule
Tables & Data elements (pixels, array values) \\
Customer orders & Input data values to process \\
Waiters & GPU threads (the workers) \\
Kitchen storage & GPU memory (where all data is stored) \\
Cooking stations & GPU cores (where processing happens) \\
Waiter's notepad & Thread's registers (personal workspace) \\
Restaurant manager & GPU scheduler (coordinates everything) \\
Table numbers & Memory addresses (location identifiers) \\
\bottomrule
\end{tabular}
\end{table}

\section*{SPMD: Same Programme, Multiple Data}

Every waiter follows this \textbf{identical} service protocol:

\begin{lstlisting}
1. Greet customers at assigned table
2. Record their specific orders
3. Submit order to kitchen with table identifier
4. Collect prepared food for assigned table
5. Serve the customers
6. Process payment transaction
\end{lstlisting}

\textbf{The Key Principle:} Identical steps, different data
\begin{itemize}
    \item Waiter 42 processes orders from Table 42
    \item Waiter 873 processes orders from Table 873
    \item All executing simultaneously
\end{itemize}

In GPU terms: Each thread executes the same kernel code but operates on different data elements.

\section*{Visual Representation: Parallel Service Timeline}

\begin{verbatim}
Time 00:00 - All waiters greet simultaneously
+----+----+----+----+----+
| W1 | W2 | W3 | W4 | W5 |  W = Waiter at table
+----+----+----+----+----+
| W6 | W7 | W8 | W9 | W10|  (Showing 15 of 1000)
+----+----+----+----+----+
| W11| W12| W13| W14| W15|
+----+----+----+----+----+

Time 00:01 - All taking orders simultaneously
+----+----+----+----+----+
| O  | O  | O  | O  | O  |  O = Order taking
+----+----+----+----+----+
| O  | O  | O  | O  | O  |
+----+----+----+----+----+
| O  | O  | O  | O  | O  |
+----+----+----+----+----+

Time 00:02 - All serving food simultaneously
+----+----+----+----+----+
| V  | V  | V  | V  | V  |  V = Service complete
+----+----+----+----+----+
| V  | V  | V  | V  | V  |
+----+----+----+----+----+
| V  | V  | V  | V  | V  |
+----+----+----+----+----+
\end{verbatim}

\section*{GPU Computing: The Numbers}

\textbf{Technical Specifications:}
\begin{itemize}
    \item Modern GPUs contain \textbf{10,000+ parallel processing threads}
    \item NVIDIA RTX 4090: 16,384 CUDA cores
    \item Processing a 4K image (8.3 million pixels): Each pixel receives dedicated processing
    \item Typical performance: Billions of operations per second
    \item Beamline application: Each X-ray detector pixel represents one \enquote{table} requiring service
\end{itemize}

\textbf{Performance Comparison:} A calculation requiring one hour on a CPU might complete in just two minutes on a GPU. This represents a 30× speed improvement through parallel processing.

\par\noindent\rule{\linewidth}{0.4pt}\par

\textbf{Key Concept:} GPUs excel when identical operations must be performed on large datasets - precisely like our restaurant scenario where every table requires the same service procedure but with unique customer orders.

\textit{This parallel processing approach is fundamental to accelerating scientific image analysis at Diamond Light Source, where millions of detector pixels must be processed rapidly.}

\end{document}
