\documentclass[11pt, a4paper]{article}
\usepackage[T1]{fontenc}
\usepackage{lmodern}
\usepackage[UKenglish]{babel}
\usepackage{csquotes}
\usepackage{graphicx}
\usepackage{hyperref}
\usepackage{amsmath}
\usepackage{amssymb}
\usepackage{booktabs}
\usepackage{longtable}
\usepackage{xcolor}
\usepackage{listings}
\usepackage{enumitem}
\usepackage[a4paper, margin=2cm]{geometry}

% remove page numbering
\pagenumbering{gobble}

% uniform spacing after punctuation
\frenchspacing

% metadata
\title{Welcome to Diamond Light Source\\[0.5em]
\large Data analysis: Scientific image processing by GPU}
\author{}
\date{}

\begin{document}

\maketitle

Welcome to Diamond Light Source, the UK's national synchrotron facility. Over the next 4 days, you will engage with computational techniques used in modern scientific research. This programme introduces you to GPU programming and its applications in processing experimental data from our beamlines.

\section*{Programme Overview}

\subsection*{Day 2: Foundations}
\begin{itemize}
    \item Introduction to GPU architecture and parallel computing principles
    \item Comparison of CPU and GPU processing capabilities
    \item Access and navigation of High-Performance Computing (HPC) systems
    \item Fundamental concepts of kernel operations and image processing
\end{itemize}

\subsection*{Day 3 (morning): GPU Programming Development}
\begin{itemize}
    \item Implementation of computational kernels using CuPy
    \item Translation of sequential algorithms to parallel GPU codes
    \item Understanding CUDA thread indexing and memory management
    \item Development of 2D image processing operations
\end{itemize}

\subsection*{Day 4: Application to Experimental Data}
\begin{itemize}
    \item Processing real scientific images from beamline experiments
    \item Implementation of noise reduction and pixel binning algorithms
    \item Performance optimisation for large-scale data processing
    \item Visualisation and analysis of processed results
\end{itemize}

\subsection*{Day 5: Presentation Preparation and Delivery}
\begin{itemize}
    \item Preparation of technical presentation
    \item Presentation to invited guests and fellow students
\end{itemize}

\section*{Learning Objectives}

By the end of this programme, you will:
\begin{itemize}
    \item Understand the fundamental principles of GPU computing
    \item Write and execute GPU-accelerated programmes
    \item Apply computational techniques to real scientific data
    \item Communicate technical concepts to different audience
\end{itemize}

\section*{Expectations and Support}

\begin{itemize}
    \item \textbf{Technical Challenges}: GPU programming is an advanced topic used in advanced scientific computing. You will encounter complex concepts and technical challenges. This is an expected part of the learning process.
\item \textbf{Inquiry and Discussion}: Questions are strongly encouraged throughout the programme. Scientific progress depends on questioning, testing, and refining our understanding.
\item \textbf{Iterative Development}: Programming involves debugging and refinement. Initial attempts may not succeed, which provides valuable learning opportunities.
\item \textbf{Practical Application}: The techniques you learn are actively used by Diamond's scientists to analyse experimental data. Your work contributes to understanding these computational methods.
\end{itemize}

\section*{Programme Resources}

You will have access to:
\begin{itemize}
    \item Prepared educational materials and code examples
    \item HPC facilities
    \item Real experimental data from Diamond's beamlines
\end{itemize}

\section*{Logistics}

Please arrive by 09:00 at the entrance of Diamond House and we will meet there, ideally with your passion in scientific computing!

\end{document}
