\documentclass[11pt, a4paper]{article}
\usepackage[T1]{fontenc}
\usepackage{lmodern}
\usepackage[UKenglish]{babel}
\usepackage{csquotes}
\usepackage{graphicx}
\usepackage{hyperref}
\usepackage{amsmath}
\usepackage{amssymb}
\usepackage{booktabs}
\usepackage{longtable}
\usepackage{xcolor}
\usepackage{listings}
\usepackage{enumitem}
\usepackage[a4paper, margin=2cm]{geometry}
\usepackage{microtype}

% remove page numbering
\pagenumbering{gobble}

% uniform spacing after punctuation
\frenchspacing

% configure listings for code
\lstset{
    basicstyle=\ttfamily\small,
    breaklines=true,
    frame=single,
    xleftmargin=2em,
    framexleftmargin=1.5em,
    showstringspaces=false,
    extendedchars=false,
    upquote=true,
    literate={--}{{-{}-}}2,
    language={}
}

% inline code command
\newcommand{\code}[1]{\texttt{#1}}

% checkbox command for checklist
\newcommand{\checkbox}{$\square$}

% document metadata
\title{Terminal Navigation Exercises: Organising Your Project\\[0.5em]
\large\textbf{Data Analysis: Scientific image processing by GPU}}
\author{}
\date{}

\begin{document}

\maketitle

\section*{Introduction}

These exercises will guide you through essential terminal commands for navigating the file system and organising your project files. Complete each task in sequence and mark them as done.

\section*{Exercise 1: Determining Your Location}

Understanding your current location in the file system is fundamental.

\begin{itemize}
    \item Open terminal and type: \code{pwd}
    \item You should see: \code{/home/<FEDID>}
    \item This is your home directory
\end{itemize}

\textbf{Learning outcome:} The \code{pwd} command displays your \textit{p}resent \textit{w}orking \textit{d}irectory.

\section*{Exercise 2: Creating Project Structure}

Establish a proper workspace for your project.

\begin{itemize}
    \item Make project folder: \code{mkdir gpu\_work}
    \item Type \code{cd gpu\_work}
    \item Verify: \code{pwd} (should show \code{/home/<FEDID>/gpu\_work})
\end{itemize}

\textbf{Learning outcome:} \code{mkdir} creates directories, \code{cd} changes your current directory.

\section*{Exercise 3: Organising Your Work}

Create a logical folder structure for the project.

\begin{itemize}
    \item Create subfolders:
\end{itemize}

\begin{lstlisting}
mkdir day2
mkdir day3
mkdir day4
\end{lstlisting}

\begin{itemize}
    \item List them: \code{ls}
    \item Make a test file: \code{nano README.txt}
    \item Type: \enquote{My GPU Work Experience Project}
    \item Save: \code{Ctrl+O}, \code{Enter}
    \item Exit: \code{Ctrl+X}
\end{itemize}

\textbf{Learning outcome:} Creating multiple directories and using the \code{nano} text editor.

\section*{Exercise 4: Directory Navigation}

Practice moving between directories and creating files in different locations.

\begin{itemize}
    \item Go to day2: \code{cd day2}
    \item Create file: \code{nano notes.txt}
    \item Add: \enquote{Today I fell in love with GPUs!}
    \item Save and exit
    \item Go back up: \code{cd ..}
    \item List everything: \code{ls -la}
\end{itemize}

\textbf{Learning outcome:} \code{cd ..} moves up one directory level, \code{ls -la} shows detailed file information including hidden files.

\section*{Exercise 5: Job Submission Practice}

Create and submit a job to the cluster with proper resource allocation.

\begin{itemize}
    \item Create: \code{nano test\_job.sh}
    \item Type:
\end{itemize}

\begin{lstlisting}
#!/usr/bin/env bash
#SBATCH --job-name=test_gpu
#SBATCH --partition=cs054r
#SBATCH --nodes=1
#SBATCH --ntasks=1
#SBATCH --time=00:05:00

echo "Hello $(whoami)"
echo 'You are running your script from the cluster.'
date
hostname
\end{lstlisting}

\begin{itemize}
    \item Submit: \code{sbatch test\_job.sh}
    \item Check status: \code{squeue -{}-me}
    \item View output: \code{cat test\_*.out}
\end{itemize}

\textbf{Learning outcome:} Creating and submitting a batch job with resource specifications.

\section*{Exercise 6: Copying Files}

Learn to duplicate files and directories.

\begin{itemize}
    \item Return to gpu\_work directory: \code{cd \textasciitilde/gpu\_work}
    \item Copy your README: \code{cp README.txt README\_backup.txt}
    \item Verify: \code{ls}
    \item Copy entire day2 folder: \code{cp -r day2 day2\_backup}
    \item List to confirm: \code{ls -la}
\end{itemize}

\textbf{Learning outcome:} \code{cp} copies files, \code{-r} flag enables recursive copying for directories.

\section*{Exercise 7: Moving and Renaming}

Practice relocating and renaming files.

\begin{itemize}
    \item Move backup file to day2: \code{mv README\_backup.txt day2/}
    \item Navigate to day1: \code{cd day2}
    \item List contents: \code{ls}
    \item Rename the file: \code{mv README\_backup.txt project\_info.txt}
    \item Verify: \code{ls}
\end{itemize}

\textbf{Learning outcome:} \code{mv} serves dual purpose for moving files and renaming them.

\section*{Exercise 8: Removing Files Safely}

Learn to delete files and directories with caution.

\begin{itemize}
    \item Create test file: \code{nano temp.txt}
    \item Add: \enquote{Temporary file for deletion}
    \item Save and exit
    \item Remove it: \code{rm temp.txt}
    \item Confirm deletion: \code{ls}
    \item Go back: \code{cd ..}
    \item Remove backup directory: \code{rm -r day2\_backup}
    \item Verify: \code{ls}
\end{itemize}

\textbf{Learning outcome:} \code{rm} \textbf{permanently} deletes files, \code{-r} flag removes directories recursively. Use with caution.

\section*{Completion Checklist}

Mark off completed tasks:

\begin{itemize}[label=\checkbox]
    \item Created project directory structure
    \item Successfully navigated between directories
    \item Created and edited files using nano
    \item Submitted first cluster job with resource specifications
    \item Viewed job output
    \item Copied files and directories
    \item Moved and renamed files
    \item Safely removed unwanted files
\end{itemize}
\end{document}
